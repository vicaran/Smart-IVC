%%%%%%%%%%%%%%%%%%%%%%%%%
\section{\applicationName} \label{projectDesign}
%%%%%%%%%%%%%%%%%%%%%%%%%
\subsection{Environment and Frameworks}
The environments selected to develop \applicationName\ are the following:
\begin{itemize}
	\item {\bf The Server--side} was written using the {\bf Java} Programming Language. The {\bf Spring Framework} was used, in particular using its most famous convention--over--configuration, called {\bf Spring Boot}. The database used to store the information about the city is {\bf MySQL}.
	\item {\bf The Client--side} was written using the {\bf Javascript} scripting language. The {\bf jQuery} library was used and, of course, {\bf HTML5} and {\bf CSS3}. As mentioned above, the 3D visualization of the city was achieved using the {\bf Cesium Framework}
\end{itemize}

\subsection{The Overall Structure}
Figure \ref{fig:babilonJS} shows how the final prototype of \applicationName\ looks like.
\begin{figure}[H]
\centering
\includegraphics[width=0.8\textwidth]{chapter3/images/project_structure}
\caption{\applicationName\ architecture scheme}
\label{fig:project_structure}
\end{figure}
The upcoming sections, illustrate \applicationName\ and how it was developed. The first part proposes the \applicationName\ modelling of data. Afterwards a section about the parser and the cron--jobs created is present. Followed by a description of the API provided to the user. Finally it concludes with some discussions about how the structure and the logic behind the Web--Application was developed.
\subsection{Server Side}
\subsubsection{Available Data and Modelling}
The entire work starts from the ``xml'' file provided by the Comune of Lugano. The first choice to take was how to model the available data in the best way such that the way of retrieving information about any building was as fast and as consistent as possible.\\
The models created were the following:
\begin{itemize}
	\item City
	\item Suburb
	\item Building
	\item Address
	\item Type
\end{itemize}
\subsubsection{Parser}
The data type taken as input is of xml type. It is used in the process to load buildings data.\\
At the beginning of the file an header which describes the content is defined. It regulates fields and types as well as values range, tolerances in the coordinates system and other set ups parameters. After the header, a long record of elements which represent buildings follows (exactly 18904). Structure and constraints, of record elements are defined in the header. The record presents several data, out of this data  \applicationName\ finds use on almost all of them. Here there will be described the tags that can be found in the xml file and that were found useful for \applicationName\:
\begin{itemize}
	\item {\bf SHAPE:} it stores both the perimeter of the building and the max bounds that it occupies
	\item {\bf Descrizione:} it represents the type of the building. Between them only the ones of type ``Edificio'' (i.e., building in Italian), are stored
	\item {\bf Sezione:} it is a numeric value that represent the suburb in which the building is located
	\item {\bf NUM$\_$CIVICO$\_$ID:} it is a 8--character--long numeric value that contains: the number id representing the street name in the first four characters and the civic number in the remaining four.
	\item {\bf EGID$\_$UCA:} it uniquely identifies buildings on the entire Swiss ground on the basis of the ``Registro Federale degli Edifici e delle abitazioni'' (REA) from the ``Ufficio federale di statistica''.
	\item {\bf PIANI:} the number of floors per building
	\item {\bf SHAPE$\_$AREA:} the area of the plane described by the building
	\item {\bf SHAPE$\_$LENGTH:} the length of the perimeter of the building
\end{itemize}
The data is assimilated in \applicationName\ through the loader. A parser has been created in order to read and parse a file structured in the way described above. The parser creates a building model and an address model for every record in the xml file and sets their fields using the information stored in the xml tags. Once the data is created, the models gets stored into a database.\\

Coordinates of buildings are stored in a data type called CH1903. It represents the Swiss projection coordinates system. It uses an Oblique Mercator on a 1841 Bessel ellipsoid. An Oblique Mercator is an oblique conformal cylinder projection. Together with the 1841 Bessel ellipsoid, a reference ellipsoid of geodesy with base at the old observatory in Bern, gives meaning to the projected coordinates. The transformation computations of the data from CH1903 to WGS84 is provided from the Swiss confederation. The precision with this transformation is of respectively 1 meter and 0.1". In order to derive the transformationa specific formulae has to be applied. The way these coordinates are converted can be found on the website of the Federal Office of Topography Swisstopo under the section NAVREF. 
\subsubsection{Commands}
On the application start, the application checks if the data is already present in the database, if not it executes a command that reads the xml file and parses it using the parser described above and stores the models in the database.\\  
At this point, the data is not ready yet, since it does not contain all the important information required from \applicationName\ to work.\\
That is why some commands where create for different purposes. Again, on application start, it is checked if the data is updated properly, if not the following commands are executed:
\begin{itemize}
	\item The {\bf CityLoaderCommand} contains two commands to be executed: the first one, as explained above, parses the xml and stores the models in the database. The second one, adds some additional information about the buildings (i.e., number of apartments--per--building that are primary and secondary houses). This additional information was received by the Comune of Lugano in a second time in the format of a txt file. Therefore, the matching between buildings listed in the two files was done using the EGID value. 
	\item The {\bf ConverterCommand} is used to convert the coordinate system used for the perimeters of the buildings from CH1903 to WGS84. The conversion is done using the service provided by the Web APIs of the Federal Office of Topography Swisstopo  where, given a pair of coordinates in the CH1903 system, their respective conversion in WGS84 is returned.
	\item The {\bf CityInformationCommand} is used to add additional information to each building though three different services. The first one adds information about the membership suburb using an APIs service provided by Swisstopo that uses the EGID in order to get such an information. Another service provided by Swisstopo API's is used to get information about the address name and the civic number of the building, still using the EGID value. The last command uses the API service provided by OpenStreetMaps in order to get information about addresses, civic numbers (for the building that have not an EGID value stored) and type of the building (e.g., Hospital, School, University etc \dots).  
\end{itemize}
Once these commands are executed, the data is ready to be used and retrieved in order to be shown on the web application.
\subsubsection{Controllers}
\subsection{Client Side}
\subsubsection{The first Attempt: Babylon.js}
Immediately after having gathered the essential data useful to draw the buildings of the city of Lugano, a first attempt of visualizing them was made using BabylonJS. It is a JavaScript framework for building 3D environments with HTML5 and WebGL. It allows the creation of a scene with customized lights, cameras, materials, meshes, animations, audio and actions. It also supports scene picking (i.e., an element on the scene is clickable and it is possible to interact with it).\\

In order to test the capabilities of BabylonJS, a small portion of Lugano (i.e., the part around the lake) was selected and rendered. The result can be seen in the following image:
\begin{figure}[H]
\centering
\includegraphics[width=0.8\textwidth]{chapter3/images/babylonJS}
\caption{The first attempt of city visualization using BabilonJS}
\label{fig:babilonJS}
\end{figure}
As long as the buildings showed in the scene where under the two--thousand, the browser was very fast in rendering during the loading of the page and the movement of the camera was smooth.\\
The main problems that make the idea to use BabylonJS discarded, were has been basically two:
\begin{itemize}
	\item The buildings to be rendered were far above the mentioned threshold, the rendering would take more than 30 seconds.
	\item BabylonJS provides no basis to start from: the initial scene is completely empty and, as it is possible to see in the image above, buildings lay on a plane. That represented a serious problem since the visualization of the city was planned to be shown in a way that would be as realistic as possible.
\end{itemize}
A solution to the last problem would have been using the Google Elevation API. This service, given a pair of coordinates (i.e., latitude and longitude), returns the exact altitude of that point.\\

Unfortunately, the API system of Google provides just 2,500 free requests per day. In the case of Lugano, that extends its territory for more that $25km^2$, considering one request for meter it would have been taken more than 10 days to get the entire terrain structure (for the entire city of Rome, that spans almost $46km^2$, the days taken would have been more than 20).\\

Nonetheless, this would have made the rendering slower since, in addition to the visualization of the buildings, also a rendering of the terrain (i.e., lakes, mountains and rivers) would have taken place.\\
Therefore, this lack of both Google--API--requests and good performances, lead the idea to use BabylonJS to be discarded.
\subsubsection{The final version: Cesium.js}
As stated above, the Cesium Framework was finally used to build the client--side of the application.\\
Cesium, on the contrary, provides a ready--to--use virtual globe
\subsubsection{The Side Menu and The Query Builder}
\subsubsection{Cesium Framework}