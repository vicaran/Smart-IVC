%%%%%%%%%%%%%%%%%%%%%%%%%
\section{Conclusion} \label{conclusions}
%%%%%%%%%%%%%%%%%%%%%%%%%
\subsection{Potential and Limits}
In our opinion the tool has great potential. As explained above, even if there are many tools which represent cities in 3D, very few of them are able to allow interactions and almost none of them allow the possibility to do query like presented in \applicationName. We think that the possibility to take plain alphanumeric data and gives it a virtual representation which provides visual feedback is the success key of this application.\\

The main limits of \applicationName\ are the lack of additional data about the urban environment (e.g., traffic data) and the quality of data provided by third party entities (e.g., OpenStreetMap). We did not manage to always find accurate data to inject in the example of Lugano, therefore we were limited to make interactions with the available data. Since not all data was available for each building, where data was missing we had to fill in with OpenStreetMap's information that were not always accurate.

\subsection{Future Work or Possible Developments}
Right now \applicationName is a prototype which could be further developed, since the ways out of this application are really limitless. With some more data available like a more accurate representation of the buildings, it could be possible to result in an even nicer and more precise 3D--visualization. In addition, more information related to buildings, bus stops, actual streets, rivers, bridges, green zones and tunnels could be included. Once other elements are available in the representation further interactions could be implemented raising the level of visual feedback provided by the tool.\\

\applicationName, with its graphical rapresentation and interactivity, has raised interest in both private citizens and the public sector. The City of Lugano stated interest on the potential application of \applicationName, and suggested a novel visualization on data that involves primary and secondary residence.\\

Our future work involves integration with better data sources, like Google Maps. We also plan to integrate more refined 3D models of buildings as provided by Swisstopo.
