%%%%%%%%%%%%%%%%%%%%%%%%%
\section{Conclusion and Future Work} \label{conclusions}
%%%%%%%%%%%%%%%%%%%%%%%%%
\applicationName\ has great potential. As explained above, even if there are many tools which represent cities in 3D, very few of them are able to allow interactions and almost none of them allow the possibility to do queries like in \applicationName. We think that the success of this application resides in the possibility to take plain alphanumeric data and give a visual and interactive representation to them which provides visual feedback.\\

The main limits of \applicationName\ are the lack of additional data about the urban environment (e.g., traffic data) and the quality of data provided by third party entities (e.g., OpenStreetMap). We did not manage to always find accurate data to integrate in the example of Lugano, therefore we were limited to make interactions with the available data. Since not all data was available for each building, where data was missing we had to fill in with OpenStreetMap's information that were not always accurate. For example, it occurred that the same type was assigned to different buildings just because they were near. This means that the service provided by OpenStreetMap is not always precise and could lead to incorrect data.

\subsection{Future Work and Possible Developments}
Right now \applicationName\ is a prototype which could be further developed, since the possible improvements of this application are really limitless. With some more data available, e.g., a more accurate representation of the buildings, we could provide an even nicer and more precise 3D--visualization. In addition, more information related to buildings, bus stops, actual streets, rivers, bridges, green zones, and tunnels could be included. Once other elements are available in the representation, further interactions could be implemented improving the level of visual feedback provided by the tool.\\

\applicationName, with its graphical representation and interactivity, has raised interest in both private citizens and the public sector. The City of Lugano stated interest on the potential application of \applicationName, and suggested a novel visualization on data that involves primary and secondary residences.\\

Our future work involves integration with better data sources, like Google Maps. We also plan to integrate more refined 3D models of buildings as provided by Swisstopo.
