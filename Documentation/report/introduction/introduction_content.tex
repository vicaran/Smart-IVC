%%%%%%%%%%%%%%%%%%%%%%%%%
\section{Introduction} \label{introduction}
%%%%%%%%%%%%%%%%%%%%%%%%%
Cities nowadays are in constant evolution. They change their structure over time through the appearance of new entities and the disappearance of old ones, thus they are very unsteady and prone to changes. Important changes might be considered with regard to the deployment and management of all types of infrastructures within cities. For example, the decision of which building to demolish in order to make space for the construction of a new mall in a city.\\

Moreover, cities have a very large impact on the economic and social development of nations: they represent the real foundation where people live, where companies have their business and in which numerous services are provided. Often, a city overview is not available, so decision may be taken without having a big picture of the surrounding environment. This could lead to choices that might have a negative impact on the current city, reducing the efficiency and the life quality of the citizens.\\

To control these changes, a visualization, which provides a city overview, is required. Such a visualization is considered to be the first step towards what it is called a Smart City. A Smart City can be defined as a city which uses information and communication technologies so that its critical infrastructure as well as its components and public services provided are more interactive, efficient and so that citizens can be made more aware of them.\\


\subsection{Contributions}
\applicationName\ aims to solve the problem generated by the static nature of data regarding cities (as mentioned above) through an Interactive 3D--Visualization model. Therefore, the main contributions that \applicationName\ provides are:
\begin{itemize}
	\item The visualization of a 3D city model. As a use case for this Bachelor Project, the city of Lugano has been taken into account.
	\item The interactivity that the user has with the entities inside the city: buildings can be clicked in order to receive information about them and in order to receive information between the selected building itself and the rest of the entities in the city.
	\item The possibility to have a graphical city overview using a very intuitive system of visual queries which produce results that are immediately visible through the highlighting or colouring of the buildings. 
\end{itemize} 
\subsection{Structure of the report}
\begin{itemize}
	\item {\bf Chapter 2} discusses the state of the art. It talks about already existing tools which allow city visualization in a virtual environment. In particular, it focuses on tools that use the Cesium framework comparing and contrasting the existing features with the ones proposed in \applicationName.
	\item {\bf Chapter 3} is the main chapter of the entire report. It shows how \applicationName\ has been developed step by step. The decisions that have been taken and the technologies adopted will be explained and discussed in detail starting from the parsing of the unique .xml file provided by the Comune of Lugano and ending with the final result (i.e., the 3D--visualization of the entire city). 
	\item {\bf Chapter 4} illustrates showing some use cases of the application. Examples will be illustrated and explained in detail, in order to show the features proposed by \applicationName. 
	\item {\bf Chapter 5} is the conclusion of this report. Here, the current limitations of the application will be presented. Finally, we will discuss and analyze the future work and the different paths that \applicationName\ might take.
\end{itemize}